\chapter{impacto}

Según estimaciones oficiales, la aplicación del ECE podría representar el ahorro de 38 mil millones de pesos para el sistema de salud, debido a que se contrarrestarían posibles negligencias médicas, retrasos en la atención, cirugías, robo y desperdicio de medicamento, entre otros. Esto debido a que la falta de información clínica retrasa la atención y puede ser la causa de errores médicos. Esta evolución tecnológica permitirá aumentar la productividad en 20 por ciento; reducir los tiempos y días de espera para consultar en 60 por ciento y ahorros de hasta el 80 por ciento en papelería; reducir los tiempos para cirugía que llegan a ser de hasta 62 días, así como disminuir el desperdicio de medicamento. Además de colocar a México a la altura de otros países que ya implementan este mecanismo. Manual del Expediente Clínico Electrónico. Dirección General de Información en Salud. Secretaría de Salud. México, 2011.
Pacientes con estabilidad respiratoria, hemodinámica y neurológica, el tiempo de espera máximo debe ser de 60 minutos; y verde, pacientes con estabilidad respiratoria, hemodinámica y neurológica, con aspecto saludable y sin riesgo evidente de complicaciones, el tiempo de espera es de hasta cuatro horas. (Chiapas, 2017)
La implementación de un expediente clínico en el estado de Chiapas representaría un gran ahorro económico, tiempo y así como agilizar la atención de cada uno de los pacientes. Actualmente el tiempo de espera para los pacientes en los hospitales llega a ser de cuatro horas, siendo este demasiado tiempo, el cual las personas pudiesen ocupar para realizar sus actividades económicas ya que en el estado un gran porcentaje de la población vive al día con lo poco que gana durante una jornada laborar, siendo esta de una ganancia no estable. Se espera reducir el tiempo de espera en un 80\% con el expediente clínico electrónico.
El impacto Tecnológico principal del proyecto será que se desarrolle a las necesidades de la secretaria de salud del estado de Chiapas, siendo una de estas que funcione de manera online y offline, además que pueda contar con una gran escalabilidad ya que es pensado para manejar grandes cantidades de información, datos estadísticos y por supuesto con la mayor seguridad posible.  Con ello se pretende eliminar en su mayoría la duplicidad de los datos o la perdida da de los mismos. Además, se ayudará a aplicar acciones preventivas en la población. Se tendrá un acceso más rápido a la información para la ayuda de investigaciones y desarrollo de la salud en el estado.
