\chapter{Introducción}
Este documento tiene como objetivo ser una guía general para utilizarla como plantilla
en trabajos como: Proyecto de Investigación, Reporte de Investigación, Tesina, o Tesis.

En este documento haremos algunas recomendaciones en la forma en como deberías estructurar
tu documento fuente.

\section{Antecedentes del Problema}

Un proyecto de investigación es un procedimiento que, siguiendo el
método científico, pretende recabar todo tipo de información y
formular hipótesis acerca de cierto fenómeno social o científico,
empleando las diferentes formas de investigación.
\vspace{20}
Son el punto de inicio para la delimitación del problema. se pueden
mencionar las experiencias individuales, materiales escritos
(libros, revistas, periódicos y tesis), teorías, descubrimientos
producto de investigaciones, conversaciones personales,
observaciones de hechos, creencias e incluso presentimientos.



\section{Planteamiento del Problema}

\section{Hipótesis}
\label{sec:hipotesis}

\section{Objetivos}

    \subsection{Objetivos Generales}
    \label{subsec:objGrales}
  Desarrollar un Sistema de Expediente Clínico Electrónico para la secretaría de salud, que se estima implementarse en todo el estado de Chiapas, para una mejor atención a los pacientes y un mejor control de los datos, evitando la duplicidad de los mismos.

    \subsection{Objetivos Específicos}
    \label{subsec:objetivos especificos}
\begin{itemize}
  \item Desarrollar un sistema offline y online.
  \item Reducir el tiempo de espera en la recolección de datos.
  \item Evitar las pérdidas de información.
  \item Evitar la duplicidad, con una buena administracion de la base de datos.

\end{itemize}


\section{Justificación}

