%\chapter{Introducción}
La presente investigación pretende desarrollar un Sistema de Expediente Clínico Electrónico (SECE) para la Secretaria de Salud de Chiapas. Se presume que el SECE agilice las actividades de captura y recolección de datos de cada paciente y evitar la duplicación de datos de cualquier paciente, tendría un impacto enorme en cuanto a la facilidad de uso, la reducción de tiempo y una mejora sorprendente en la atención en el paciente. La Secretaria de Salud ha visto la necesidad de crear un tipo de solución para satisfacer las necesidades de los trabajadores y pacientes que son cada vez más exigentes con la innovación y la tecnología.

\section{Antecedentes del Problema}
Actualmente la Secretaria de Salud  cuenta con 10 hospitales, los cuales son: Hospital General Regional, Dr. Rafael Pascasio Gamboa, Clínica Hospital del ISSSTE, Hospital del ISSSTE, Hospital de Especialidades del ISSTECH, Hospital General del IMSS, Zona I, Hospital General del IMSS, Zona No. II, Clínica Hospital de Tapachula-ISSTECH, Clínica de Consulta Externa del ISSTECH, Delegación Estatal de la Cruz Roja, Coordinación Médico de la Cruz Roja, de las cuales atienden a más de 1,000 de personas diariamente, lo cual hace muy pesado las labores de registro de los paciente, ya que se hace todo en formato de papel. Uno de los mayores problemas que presenta la Secretaria de Salud es la duplicidad de datos y la perdida de los mismos, hace algunos años compraron un sistema de expediente clínico electrónico que les costó mucho dinero, el cual, hoy en día no lo utilizan por la falta de datos y el consumo relevante de tiempo en que se tomaban en capturar los datos, por lo cual optaron nuevamente regresar a los documentos en papel.
Primero, definiremos el concepto de expediente; recopilación de información detallada y ordenada cronológicamente con relación a la salud de un paciente en un período determinado. Representa el cimiento para conocer las condiciones de salud, aspectos médicos y procedimientos realizados por profesionales de la salud. \cite{Gracia}
En fechas reciente, las autoridades sanitarias tomaron la decisión de sustituir el modelo de atención médica tradicional apoyándose en un Sistema de Expediente Clínico Electrónico que aportará múltiples ventajas para la institución y al paciente. Además de facilitar la captura de datos al paciente reduciendo tiempo y brindando una mejor atención al paciente. Nuestro Sistema de Expediente Clínico Electrónico será online y offline el cual permitirá tener siempre los datos más cerca, aunque el servidor sufra de fallas, no se perderán los datos. Se afirma que el cambio brindará seguridad en la provisión del cuidado de salud y mejorará la calidad de la atención, ya que, de no disponer de la información clínica esencial en el momento en que se necesite, se considera una de las fuentes principales de errores de los profesionales de la salud, los que pueden prevenirse mediante el acceso a un Sistema de Expediente Clínico Electrónico disponible en línea para su consulta en cualquier nivel de atención del sector de Salud. Para normalizar y homologar las funcionalidades, garantizar la interoperabilidad, procesamiento, interpretación, confidencialidad, seguridad y uso de estándares y catálogos de la información de los registros electrónicos en salud, la Secretaría de Salud publicó la Norma Oficial Mexicana NOM-024-SSA3-2010.  “Que establece los objetivos funcionales y funcionalidades que deberán observar los productos de Sistemas de Expediente Clínico Electrónico para garantizar la interoperabilidad, procesamiento, interpretación, confidencialidad, seguridad y uso de estándares y catálogos de la información de los registros electrónicos en salud “\cite{SEGOB}
Las ventajas que impactarán la calidad de la atención habrá mejoras evidentes en la gestión administrativa; sin embargo, la introducción de nuevas tecnologías de la información exige tomar medidas para prevenir que la información de la base de datos se utilice por personas no autorizadas, y que ello redunde en perjuicio de los derechos del paciente; lo que no se ha regulado en México de forma eficiente, pues la normatividad en materia de protección de datos personales (datos sensibles) en poder de instituciones públicas, empresas privadas y particulares en general no es acorde con los estándares internacionales. México está inmerso en una sociedad globalizada que demanda comunicación y que en todo momento intercambia y comparte información. Ante esta realidad, han surgido nuevas tecnologías, herramientas y lineamientos que facilitan mediante su implementación y uso, el fortalecimiento de la sociedad de la información. El Sector Salud no es ajeno a esta realidad y ha identificado en las tecnologías de la información y las telecomunicaciones, un aliado para aumentar la eficiencia y mejorar la calidad en la prestación de cuidados de la salud redundando en un mayor bienestar de la población. En este ámbito se presenta un instrumento, el expediente clínico electrónico (ECE), el cual permite asegurar que los pacientes reciban el más oportuno, conveniente y eficiente cuidado de la salud. El ECE es una herramienta que ofrece información sobre medicación, la historia del paciente, los protocolos clínicos y recomendaciones de estudios específicos; genera un incremento en la eficiencia en el rastreo de antecedentes clínicos y el cuidado preventivo; y contribuye a reducir las complicaciones incluyendo los errores en la medicación.


\section{Planteamiento del Problema}
En la Secretaria de salud del estado de Chiapas ocurre demasiadas pérdidas de datos en el proceso de los registros de los pacientes, ocasionando redundancia, pérdida de tiempo y una mala interoperabilidad con los formatos impresos, así mismo provocando una mala atención hacia los pacientes, los cuales no son atendidos como debería de ser, al usar un sistema de expediente clínico electrónico que no cumple con sus expectativas, dedican demasiado tiempo en registrar a un paciente dejando a un  lado el cuidado del mismo .

\section{Hipótesis}
Con la implementación del Expediente Clínico Electrónico. Se logrará reducir el tiempo de atención de cada paciente, con la interoperabilidad mejoraremos el manejo de los datos en los demás centros de salud y evitará la duplicidad y la falta de seguimiento.

\label{sec:hipotesis}

\section{Objetivos}

    \subsection{Objetivos Generales}
    \label{subsec:objGrales}
  Desarrollar un Sistema de Expediente Clínico Electrónico para la secretaría de salud, que se estima implementarse en todo el estado de Chiapas, para una mejor atención a los pacientes y un mejor control de los datos, evitando la duplicidad de los mismos.

    \subsection{Objetivos Específicos}
    \label{subsec:objetivos especificos}
\begin{itemize}
  \item Desarrollar un sistema offline y online.
  \item Reducir el tiempo de espera en la recolección de datos.
  \item Evitar las pérdidas de información.
  \item Evitar la duplicidad, con una buena administracion de la base de datos.

\end{itemize}


\section{Justificación}
\label{sec:justificacion}
Con el sistema de Expediente Clínico Electrónico reduciremos un 60\% el tiempo de recolección de datos, así mismo evitará la pérdida y duplicidad de datos en un 100\% gracias a la base de datos offline que se sincronizará con la base de datos online, dando así como resultado poder consultar información esencial en el momento deseado.
