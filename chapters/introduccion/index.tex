\chapter{Introducción}
Este documento tiene como objetivo ser una guía general para utilizarla como plantilla
en trabajos como: Proyecto de Investigación, Reporte de Investigación, Tesina, o Tesis.

En este documento haremos algunas recomendaciones en la forma en como deberías estructurar
tu documento fuente.

\section{Antecedentes del Problema}

Un proyecto de investigación es un procedimiento que, siguiendo el
método científico, pretende recabar todo tipo de información y
formular hipótesis acerca de cierto fenómeno social o científico,
empleando las diferentes formas de investigación.
\vspace{20}
Son el punto de inicio para la delimitación del problema. se pueden
mencionar las experiencias individuales, materiales escritos
(libros, revistas, periódicos y tesis), teorías, descubrimientos
producto de investigaciones, conversaciones personales,
observaciones de hechos, creencias e incluso presentimientos.



\section{Planteamiento del Problema}

\section{Hipótesis}
\label{sec:hipotesis}

\section{Objetivos}

    \subsection{Objetivos Generales}
    \label{subsec:objGrales}
    El objetivo general de este documento es:
    Proporcionar al alumno una plantilla que le sirva para presentar su reporte de investigación
    para la materia de Taller de Investigación II.
    
    \subsection{Objetivos Específicos}


\section{Justificación}

