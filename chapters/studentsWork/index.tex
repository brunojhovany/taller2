\chapter{Ejemplo de Citas y Pie de Página}
\textbf{Abstract}
In this document we describe the project of the Service Oriented Architecture, first we mentioned the definitions about the distributed systems and the Web service as a little introduction for the activity. After that we describe the development of the system called “Agencia de diseño” with the problems during the activity. At the end the conclusions are mentioned.

\section{Introducción}

Ejemplo de esta parte va la Introducción ...



\section{Dos ejemplos de citas}

\subsection{Cita de libros}
Ejemplo de una cita, \cite{Scambray00}.
Este es otro ejemplo de otro libro,\cite{Kozierok05}.



\section{Otros dos ejemplos de cita}

\subsection{Cita de tesis}
Aviary es un SDK que sirve para poder ingresar un modelo de edición de imágenes. El editor Aviary está optimizado para las últimas versiones de Chrome, Firefox, Safari e Internet Explorer (IE9). Para IE8 se configura  con sólo unas pocas líneas de Javascript, \cite{Lopez05}.

\subsection{URL cita}
Si quieres citar una url puede servirte esta cita, \cite{urlCiteVilar2013}.

\section{Pie de Página}

Un pie de página es una referencia al texto que puede profundizar más sobre el tema o puede dar una explicación adicional para que el lector
pueda entender mejor la lectura.

Un ejemplo es como el siguiente:\footnote{El pie de página usualmente aparece en la parte de abajo de una página.}

Si estás hablando de HTML\footnote{Hypertext Markup Language} quizá decir quién lo fundó no sea relevante ó donde puedes saber más vía una URL\footnote{Puedes saber más de HTML vía \url{https://en.wikipedia.org/wiki/HTML}}, sin embargo y dependiendo del discurso puede servirte que el lector sepa quién lo fundó
o qué significan las iniciales.
